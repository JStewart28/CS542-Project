\documentclass{article}
\usepackage{graphicx} % Required for inserting images
\usepackage{todonotes}
\usepackage[colorlinks=true, allcolors=blue]{hyperref}

\newcommand\todoJason[1]{\todo[author=Jason,color=yellow,inline]{#1}}
\newcommand\todoMichael[1]{\todo[author=Michael,color=green,inline]{#1}}
\newcommand\todoChris[1]{\todo[author=Chris,color=orange,inline]{#1}}

\title{TODO}
\author{Jason Stewart\\
\textit{jastewart@unm.edu} 
\and
Michael Servilla\\
\textit{chico@unm.edu}
\and
Christopher Leap\\
\textit{\todoChris{email}}
\date{\today}
}

\begin{document}

\maketitle

\section{Introduction}
Question: How does the compute and communication power of personal laptops compare to that of supercomputers? In theory, supercomputers are just a bunch of normal computers connected together by a fast network. Hypothesis: We think the times of these benchmarks on a single processor should be comparable between supercomputers and our laptops. What we are doing:
\begin{itemize}
    \item Get some low-memory primarily compute-based and communication-based benchmarks (Does Dr. Bienz have any?)
    \item Run these benchmarks on a single node on different supercomputers and on our personal laptops.
    \item Do these benchmarks perform better on the supercomputers or our laptops, and does this make sense given the processor?
    \item  What if we oversubscribe a faster processor? Will our benchmarks run faster on an oversubscribed faster processor or a slower, not-oversubscribed processor?
    \item What if we run our benchmarks with one process per node so they have to communicate over the network? How does this compare to speeds analyzed previously? Are laptops faster?
\end{itemize}

\todoMichael{Do you have anything to add to our methods, question, or hypothesis?}
\todoChris{Do you have anything to add to our methods, question, or hypothesis?}

example cite: \cite{cfb_db}


\newpage
\nocite{*}
\bibliographystyle{acm}
\bibliography{references}

\end{document}

Please upload a 2 page, single space, PDF document containing the following information:
- Proposal title
- Team members
Sections:
Introduction. Describe what is the problem that you want to address, why is it important and why it qualifies as a Big Data problem
Data: describe the data that you will use, its source(s), and what is your plan to secure the data in a timely manner. Use at least two data sources. 
Methodology: explain in broad terms the approach or approaches you may consider to solve your problem. This section does not need to be very accurate at this point, you can figure out other approaches as the course continues, but I just want some ideas. 
Contingency plan: if everything fails, what is your plan B? again, it does not need to be accurate or flushed out, I just want to see that you are considering all the angles








\todoBoth{Poke around at databases so we can get data}
\todoMark{ETL Composite DB}
\todoJason{fill out proposed metrics by Friday}




We can pull this data out of the pro database, but it will require writing good queries because it's not readily available to us.


\missingfigure{put some graphs here}


